\documentclass[a4paper,12pt]{article}
\usepackage{graphicx} 
\usepackage[utf8]{inputenc}
\usepackage{polski}
\usepackage{float}
\usepackage{wrapfig} 
\input{epsfx}
\setlength{\parindent}{20pt}

\begin{document}


\title{PRACA DYPLOMOWA INŻYNIERSKA}
\maketitle

\newpage
\section*{Streszczenie}
\section*{Abstract}

\newpage 
\section{Wstęp}
	\subsection{Motywacja wyboru pracy}
 		\paragraph{\noindent} 
 		Obecnie coraz częściej w kościołach czy świątyniach możemy spotkać rzutniki czy wyświetlacze led, na których ekranizowane są teksty pieśni.
		Zbiory pieśni dostarczane przez producentów wyświetlaczy nie są w stanie sprostać dynamicznemu rozwojowi muzyki - powstawaniu nowych utworów
		oraz bardzo zróżnicowanym tradycjom lokalnym oraz indywidualnym upodobaniom muzyków kościelnych wykorzystującym różne śpiewniki.
		Czasochłonny proces budowy wspomnianych zbiorów często opiera się na ręcznym przepisywaniu tekstów ze śpiewników.
		Automatyzacja tego procesu pozwoli na łatwiejsze dopasowanie zbiorów do potrzeb muzyków kościelnych i tradycji lokalnych. 

	\subsection{Cel pracy}
		\paragraph{\noindent}
		Podjęte wysiłki miały na celu zaprojektowanie algorytmów oraz budowę aplikacji okienkowej o prostym i przyjaznym dla użytkownika interfejsie, która umożliwi pozyskanie ze zdjęć śpiewników tekstów pieśni.
		Opierając się na przetwarzaniu obrazów (detekcji i usunięciu znaków muzycznych) i rozpoznawaniu ciągów znaków abecadła (słów) uwzględniając polskie znaki diakrytyczne.

\newpage 

\section{Wykorzystane technologie}

\subsection{Język programowania Java SE 8}
\textbackslash\textbackslash wstep o jezyku \\
\textbackslash\textbackslash popularnosc co pomoze w utrzymaniu projektu \\
\textbackslash\textbackslash lambdy i strumienie ulatwiajace przetrwarzane

\subsection{Biblioteka OpenCv}

OpenCv jest biblioteką służącą do komputerowego przetwarzania obrazów oraz uczenia maszynowego, 
o otwartym kodzie źródłowym.
\subsubsection{Zarys historyczny}
Prace nad budową tej biblioteki rozpoczął jeden z pracowników firmy Intel - Gary Rost Bradski, zainspirowany środowiskiem akademickim, które wówczas posiadało bardzo bogatą infrastrukturę służącą do przetwarzania obrazów, jednak przeznaczoną dla użytku wewnętrznego. 
Studenci w obrębie jednej jednostki akademickiej dzielili się kodem zawierającym gotowe implementacje algorytmów, co znacząco ułatwiało prace przy własnych projektach czy aplikacjach. 
Stąd głównym celem biblioteki OpenCv jest udostępnienie wszystkim zainteresowanym zagadnieniami przetwarzania obrazów i sztucznej inteligencji gotowej jednolitej, darmowej infrastruktury pozwalającej na pracę, tak aby nie trzeba było ponownie ''wynajdywać koła''.\\

OpenCv zostało przedstawione po raz pierwszy w 1999 roku szerszemu gronu odbiorców. 
\paragraph{Wersja 1.0}
W wersji 1.0 kod biblioteki stanowiły wyłącznie najbardziej użyteczne przy przetwarzaniu obrazów
algorytmy zaimplementowane w języku C. 
Od tego czasu biblioteka znacząco się zmieniła. 
\paragraph{Wersja 2.0}
W wersji 2.0 znaczący wpływ na wydanie wywarły trendy obecne w prowadzeniu projektów, w których wytwarza się oprogramowanie - repozytorium kodu w systemie kontroli wersji Git, gdzie możemy znaleźć najnowszą wersję biblioteki i najświeższe poprawki, testy jednostkowe czy booty ciągłej kompilacji. 
Zaimplementowano również interfejsy dla języków programowania takich jak C++ oraz Javy, Python’a, MATLAB-a.
Od tego czasu nowe typy danych i funkcje/metody implementowano w C++, a już napisane w języku C dopasowano do nowej technologii.
\paragraph{Wersja 3.0}
Nieustannie zwiększająca się liczba zaimplementowanych algorytmów użytecznych przy przetwarzaniu obrazów przyczyniła się do modułowej budowy biblioteki w wersji 3.0, która przedstawia się w sposób następujący:


\begin{enumerate}
	\item (Warstwa wierzchnia) system operacyjny,
	\item interfejsy dla różnych języków i aplikacje,
	\item moduł \textit {opencv\_contrib} zawierający kod napisany i dołączony do biblioteki przez użytkowników,
	\item rdzeń OpenCv,
	\item optymalizacje sprzętowe (warstwa HAL \textit {ang. hardware acceleration layer}).
\end{enumerate} 

%\begin{figure}
%	\includegraphics[width=15cm,height=9cm]{budowaModulowa.png}
%	\caption{Buwowa modułowa OpenCv}
%\end{figure}

\begin{figure}[!ht]   %Srodowisko do umieszczania rysunkow
	\begin{center}
		\includegraphics[width=10cm] {budowaModulowa.png} 
	\end{center}
	\caption
	[Budowa modułowa OpenCv]  % Podpis - do spisu tresci
	{Budowa modułowa OpenCv}  %Podpis
\end{figure}

OpenCv w tej wersji wspiera budowanie i dołączanie do biblioteki walsnych modulów.

\newpage
\begin{figure}[!ht]   %Srodowisko do umieszczania rysunkow
	\begin{center}
		\includegraphics[width=7cm] {osCzasu.png} 
	\end{center}
	\caption
	[Rozwuj projektu w czasie]  % Podpis - do spisu tresci
	{Rozwuj projektu w czasie}  %Podpis
\end{figure}

 
\subsubsection{Popularność projektu}  
Projekt cieszy się bardzo dużym powodzeniem, które nieustannie rośnie. Świadczy o tym liczba pobrań 
wynosząca około 14 milionów razy, zaś miesięcznie około 200 tysięcy.
Obecnie biblioteka zawiera ponad 2500 zoptymalizowanych algorytmów, które służą do przetwarzania obrazów
również w czasie rzeczywistym i uczenia maszynowego, co znacząco ułatwia tworzenie aplikacji użytkownikom. 
Programiści piszący kod uwzględniali wymóg przenośności OpenCv, a więc możliwości kompilacji na każdym odpowiednim kompilatorze języka C++, co wymusiło restrykcyjną zgodność ze standardem i ułatwiło obsługę różnych platform.
Biblioteka dostępna jest na systemach operacyjnych takich jak: Windows, Linux, Mac OS X, do których dołączyły systemy operacyjne platforma mobilnych (Android i IOS), co znacząco przyczyniło się do zwiększenia liczby użytkowników\\

\subsubsection{Zastosowania biblioteki OpenCv}
Dynamiczny rozwój technologiczny otwiera przed biblioteką nowy horyzont użyteczności.
OpenCv znajduje zastosowanie w wielu obszarach życia do tego stopnia że wykorzystanie projektu wydaje się rzeczą całkowicie naturalna a wręcz niezauważalną. Biblioteka wykorzystywana jest w:
\begin{itemize}
	\item skanowaniu kodów QR,
	\item monitoringu,
	\item rozpoznawaniu dzwięków i muzyki,
	\item obrazowaniu medycznym,
	\item robotyce,	
	\item przemyśle - produkcji masowej i kontroli jakości,
	\item wojsku - bezzałogowe pojazdy, fotografie lotnicze,
	\item analizie obiektów,
	\item Google Street View,
\end{itemize}

\subsubsection{Wybrane funkcje i struktury danych biblioteki OpenCv}
\paragraph{Struktury danych}
\paragraph{Funkcje}

\subsection{Formaty obrazow}
\textbackslash\textbackslash
obrazy kolorowe i w skali szrosci\\
\textbackslash\textbackslash
jpg jak przechowuwany w pamieci budowa pixeli bitowo

\subsection{Eclipse}
\textbackslash\textbackslash
srodowisko i kofiguracja



\end{document}